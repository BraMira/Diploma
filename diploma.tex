\documentclass[12pt,a4paper]{amsart}
\usepackage[slovene]{babel}
%\usepackage[cp1250]{inputenc}
%\usepackage[T1]{fontenc}
\usepackage[utf8]{inputenc}
\usepackage{amsmath,amssymb,amsfonts}
\usepackage{url}
%\usepackage[normalem]{ulem}
\usepackage[dvipsnames,usenames]{color}
\textwidth 15cm
\textheight 24cm
\oddsidemargin.5cm
\evensidemargin.5cm
\topmargin-5mm
\addtolength{\footskip}{10pt}
\pagestyle{plain}

\overfullrule=15pt % oznaci predogo vrstico

\theoremstyle{definition}
\newtheorem{definicija}{Definicija}[section]
\newtheorem{primer}[definicija]{Primer}
\newtheorem{opomba}[definicija]{Opomba}


\theoremstyle{plain}
\newtheorem{lema}[definicija]{Lema}
\newtheorem{izrek}[definicija]{Izrek}
\newtheorem{trditev}[definicija]{Trditev}
\newtheorem{posledica}[definicija]{Posledica}

\newcommand{\Co}{\operatorname{Co}} %konveksna ogrinjača

\newcommand{\R}{\mathbb R}
\newcommand{\N}{\mathbb N}
\newcommand{\Z}{\mathbb Z}
\newcommand{\C}{\mathbb C}
\newcommand{\Q}{\mathbb Q}

\newcommand{\abs}[1]{ \left\lvert#1\right\rvert} 
\newcommand{\norm}[1]{\left\lVert#1\right\rVert}
% vstavi svoje definicije ...




\begin{document}


\thispagestyle{empty}
\noindent{\large
UNIVERZA V LJUBLJANI\\[1mm]
FAKULTETA ZA MATEMATIKO IN FIZIKO\\[5mm]
%\textcolor{Red}
{Finančna matematika} -- 1.~stopnja}
\vfill

\begin{center}{\large
Mirjam Pergar\\[2mm]
{\bf Računanje izotropnih vektorjev}\\[10mm]
Delo diplomskega seminarja\\[1cm]
Mentor: izred. prof. dr. Bor Plestenjak}
\end{center}
\vfill

\noindent{\large
Ljubljana, 2016} %letnica diplome
\pagebreak

\thispagestyle{empty}
\tableofcontents
\pagebreak

\thispagestyle{empty}
\begin{center}
{\bf Računanje izotropnih vektorjev}\\[3mm]
{\sc Povzetek}
\end{center}
% tekst povzetka v slovenscini

\vfill
\begin{center}
{\bf The computation of isotropic vectors}\\[3mm]
{\sc Abstract}
\end{center}
% tekst povzetka v anglescini

\vfill\noindent
{\bf Math. Subj. Class. (2010):}   \\[1mm]
{\bf Ključne besede:}   \\[1mm]
{\bf Keywords:}
\pagebreak

%1. poglavje

\section{Uvod}
\subsection{Problem}

Za dano nesingularno, kvadratno $n \times n$ matriko $A$ z realnimi ali kompleksnimi elementi, nas zanima izračun enotskega vektorja $b$ z realnimi ali kompleksnimi elementi, tako da velja:
\begin{equation}\label{eq:zac}
b^\ast Ab=0.
\end{equation}
Vektor $b$, za katerega velja \eqref{eq:zac} in $b^\ast b=1$  imenujemo \textbf{izotropni vektor}. 
%V tem problemu, je neničeln vektor skaliran in rotiran v ortogonalni smeri. %\cite{lipkin}
Bolj splošen je problem inverzne zaloge vrednosti, kjer iščemo enotski vektor $b$, za katerega velja:
\begin{equation}\label{eq:splosno}
b^\ast Ab=\mu,
\end{equation}
kjer je $\mu$ dano kompleksno število. Očitno je, da je problem \eqref{eq:splosno} možno prevesti na problem \eqref{eq:zac} za drugo matriko, saj je \eqref{eq:splosno} enako
$$b^\ast (A-\mu I)b=0.$$
Če je $\mu$ lastna vrednost matrike $A$, torej velja $Av=\mu v$, kjer je $v$ pripadajoči lastni vektor matrike $A$, potem je rešitev pripadajoč lastni vektor matrike $A$. Če pa $\mu$ ni lastna vrednost matrike $A$, je $A-\mu I$ nesingularna in je potreben izračun izotropnega vektorja te matrike. Tudi če matrika $A$ ni realna imamo opravka s kompleksno matriko, ko je $\mu$ kompleksno število. Zato bomo od sedaj naprej vse vrednosti enačili z $0$.

\begin{definicija}
Zaloga vrednosti matrike $A \in \C^{n\times n}$ je zaprta, konveksna podmnožica kompleksne ravnine, definirana kot
$$W(A)=\{x^\ast Ax: x \in \C^n, x^\ast x=1\}.$$
\end{definicija}
Očitno je zaloga vrednost $W(A)$ množica vseh Rayleighovih kvocientov matrike $A$. Če hočemo, da ima \eqref{eq:zac} vsaj eno rešitev, mora biti izhodišče v $W(A)$. $W(A)$ si lahko predstavljamo kot preslikavo, ki slika iz $n\times n$ kompleksnih matrik v podmnožico kompleksne ravnine. Označimo s $\sigma(A)$ množico vseh lastnih vrednosti matrike A, ki jo imenujemo spekter. Lastnosti zaloge vrednosti (\cite{num}):
\begin{enumerate}
\item $W(A)$ je konveksna, zaprta in omejena.
\item $\sigma(A)\subseteq W(A).$
\item Za vsako unitarno matriko $U$ je $W(U^\ast AU)=W(A).$
\item $W(A+zI)=W(A)+z$ in $W(zA)=zW(A)$ za vsako kompleksno število $z$.
\item Rob zaloge vrednosti $W(A), \partial W(A)$ je kosoma algebrska krivulja, in vsaka točka v kateri $\partial W(A)$ ni diferenciabilna je lastna vrednost matrike $A$.
\item Če je $A$ normalna, potem $W(A)=\Co(\sigma(A))$, kjer s $\Co$ označimo zaprto konveksno ogrinjačo množice.
\item $W(A)$ je daljica na realni osi, če in samo če je $A$ hermitska.
\end{enumerate}
\begin{proof}
v nastajanju...
\end{proof}
\begin{izrek}
Naj imata $A$ in $b$ realne ali kompleksne elemente. Potem veljajo enakosti:
$$b^\ast Ab=0\Leftrightarrow b^\ast (A+A^\ast)b=0 \ \textrm{in}\  b^\ast(A-A^\ast)b=0.$$
\end{izrek}
\begin{proof}
($\Rightarrow$) Če velja $b^\ast Ab=0$, je tudi $(b^\ast Ab)^\ast=b^\ast A^\ast b=0$. Če preoblikujemo prvo enačbo na desni v $b^\ast Ab +b^\ast A^\ast b$ dobimo 0. Drugo enačbo dokažemo na podoben način.\\
($\Leftarrow$) S seštevkom enačb na desni dobimo enačbo na levi:
\begin{align*}
 b^\ast (A+A^\ast)b+b^\ast(A-A^\ast)b=0\\
b^\ast (A+A^\ast+A-A^\ast)b=0\\
b^\ast (2A)b=0\\
b^\ast Ab=0
\end{align*}
\end{proof}

Če velja le $b^\ast (A+A^\ast)b=0$, ugotovimo da je $\Re(b^\ast Ab)=0$. Podobno, če velja samo $b^\ast(A-A^\ast)b=0$, potem je $\Im(b^\ast Ab)=0$. Ta dejstva bomo uporabili pri računanju rešitev za kompleksne matrike. 
Ko sta $b$ in $A$ realna, je problem mnogo enostavnejši, saj moramo upoštevati le simetričen del matrike $A$.
\begin{lema} \cite{lipkin}
Izotropni vektorji matrike A so identični izotropnim vektorjem njenega simetričnega dela.
\end{lema} 
\begin{proof}
To sledi iz $b^T Ab=b^T A_{sim} b +b^T A_{psim} b=b^T A_{sim} b,$ kjer je z $A_{sim}$ označen simetrični del matrike $A$ (t.j. $A=A^T$) in z $A_{psim}$ poševno-simetrični del matrike A (t.j. $A=-A^T$).
\end{proof}
Velja enakost:
$$b^T Ab=0 \Leftrightarrow b^T (A+A^T)b=0.$$ Hermitski in poševno-hermitski del matrike $A$ bomo označili s $H=(A+A^\ast)/2$ in $\tilde{K}=(A-A^\ast)/2=\imath K$.

\subsection{Uporaba}
Zanimanje za izračun izotropnih vektorjev je povezano s pre\-u\-če\-va\-njem delne stagnacije GMRES algoritma za reševanje linearnih sistemov z realnimi matrikami. Zaloga vrednosti se uporablja za preučevanje konvergence nekaterih iterativnih metod za reševanje linearnih sistemov in ima mnogo aplikacij v numerični analizi, diferencialnih enačbah, teoriji sistemov itd.

%2. poglavje

\section{Realne matrike}
V tem razdelku bomo opisali kako izračunamo željeno število izotropnih vektorjev za realne matrike. To storimo z uporabo lastnih vektorjev matrike $H=(A+A^\ast)/2$.\\
Ko je $A$ realna matrika, nas zanima kako izračunati rešitev naslednje enačbe:
\begin{equation}\label{eq:realna}
b^\ast Hb=0,
\end{equation}
kjer je $H$ realna in simetrična matrika (t.j. $H=H^T$). Vemo, da  je $W(A)$ simetrična glede na realno os in, da je $0 \in W(A)$, če in samo če $\lambda_n\le0\le\lambda_1$, kjer sta $\lambda_n$ in $\lambda_1$ najmanjša in največja lastna vrednost matrike $H$. Naj bosta $x_1$ in $x_n$ realna lastna vektorja, pripadajoča $\lambda_1$ in $\lambda_n$.  Potem sta $x_1^T Ax_1=x_1^T Hx_1=\lambda_1$ in $x_n^T Ax_n=x_n^T Hx_n=\lambda_n$ realni točki na skrajni levi in skrajni desni zaloge vrednosti $W(A)$ na realni osi.
Realne rešitve \eqref{eq:realna} izračunamo z uporabo lastnih vektorjev matrike $H$. Predpostavimo, da iščemo vektorje $b$ z normo 1. Matriko $H$ lahko zapišemo kot $$H=X\Lambda X^T,$$ kjer je $\Lambda$ matrika, ki ima na diagonali lastne vrednosti $\lambda_i$, ki so realna števila. $X$ je ortogonalna matrika lastnih vektorjev, tako da $X^T X=I$. Potem uporabimo ta spektralni razcep v \eqref{eq:realna}: $$b^\ast Hb=b^\ast X\Lambda X^T b=0.$$ Označimo s $c=X^Tb$ vektor projekcije $b$ na lastne vektorje matrike $H$. Dobimo naslednji izrek.
\begin{izrek} \label{izrek2}
Naj bo $b$ rešitev problema \eqref{eq:realna}. Potem vektor $c=X^T b$ s komponentami $c_i$ zadošča naslednjima enačbama:
\begin{align}
\sum_{i=1}^{n} \lambda_i \abs{c_i}^2=0 \label{eq:en1}\\
\sum_{i=1}^{n}\abs{c_i}^2=1. \label{eq:en2}
\end{align}
\end{izrek}
\begin{proof}
Enačbo \eqref{eq:en1}  dokažemo tako, da $c=X^Tb$ oz. $c^\ast =b^\ast X$ vstavimo v \eqref{eq:realna} in dobimo $$b^\ast Hb=b^\ast X\Lambda X^T b= c^\ast \Lambda c=0.$$ Ker je $\Lambda$ diagonalna matrika, lahko $c^\ast \Lambda c$ zapišemo kot vsoto komponent $\bar{c_i}\lambda_i c_i=\lambda_i\abs{c_i}^2$, ko $i=1,2,...n$.
Za enačbo \eqref{eq:en2} vemo, da je $\norm{b}_2=1$. Če normo zapišemo s $c$ dobimo $$\norm{b}_2=\norm{Xc}_2=\norm{c}_2=1,$$ saj je $X$ ortogonalna matrika.

\end{proof}
%\begin{bmatrix}
%$\lambda_1$ & &\\
% &\ddots& \\
% & &$\lambda_n$
%\end{bmatrix}
Opomba: Enačbi veljata samo za realna števila. Zaradi  izreka \ref{izrek2} mora biti 0 konveksna kombinacija lastnih vrednosti $\lambda_i$. Kot smo že videli, to pomeni, da če je $A$ definitna matrika (pozitivno ali negativno), potem \eqref{eq:realna}
nima netrivialne rešitve. Drugače je $0 \in W(A)$ in lahko vedno najdemo realno rešitev. V bistvu kadar je $n>2$,  imamo neskončno rešitev. 
\subsection{Iskanje izotropnih vektorjev}
Najprej bomo za izračun uporabili dva lastna vektorja, pozneje pa bomo videli kako se izračuna več rešitev z uporabo treh lastnih vektorjev. Če predpostavimo, da nimajo vse lastne vrednosti $H$ enakega predznaka, potem more za najmanjšo lastno vrednost $\lambda_n$ veljati $\lambda_n <0$. 
 Naj bo $k<n$ tak, da je $\lambda_k >0$ in $t$ naj bo pozitivno realno število, manjše od 1.  Označimo $\abs{c_n}^2 =t$,$\abs{c_k}^2=1-t$ in $c_i \not =0, i\not=n,k$, kar velja zaradi enačbe \eqref{eq:en2}, $t+ (1-t)=1$. Iz \eqref{eq:en1} mora veljati enačba: $$\lambda_n t +\lambda_k (1-t)=0,$$ katere rešitev je:
\begin{equation}
t_s=\frac{\lambda_k}{\lambda_k -\lambda_n}.
\end{equation}
Ker je $\lambda_n <0$, je imenovalec pozitiven in $t_s$ pozitiven ter $t_s <1$. Absolutna vrednost $c_n$ (oz. $c_k$) je kvadratni koren od $t_s$ (oz. $1-t_s$). Ker je $b=Xc$, sta dve realni rešitvi: $$b_1=\sqrt{t_s}x_n +\sqrt{1-t_s}x_k,\quad b_2=-\sqrt{t_s}x_n+\sqrt{1-t_s}x_k,$$ kjer sta $x_n$ in $x_k$ lastna vektorja, ki pripadata lastnima vrednostima $\lambda_n$ in $\lambda_k$. Druge možnosti za predznak dajo rešitve, ki so v isti smeri kot ti dve. Ker imata izraza v rešitvah enaka imenovalca, lahko rešitvi zapišemo kot: $$b_1=\sqrt{\lambda_k}x_n+\sqrt{\abs{\lambda_n}}x_k, \quad b_2=-\sqrt{\lambda_k}x_n+\sqrt{\abs{\lambda_n}}x_k$$(sledi iz \cite{lipkin}).  Vektor mora biti normiran.
%NORMIRANI
 $$b_1=\sqrt{\frac{\lambda_k}{\lambda_k +\abs{\lambda_n}}}x_n + \sqrt{\frac{\abs{\lambda_n}}{\lambda_k +\abs{\lambda_n}}}x_k,\quad b_2=-\sqrt{\frac{\lambda_k}{\lambda_k +\abs{\lambda_n}}}x_n + \sqrt{\frac{\abs{\lambda_n}}{\lambda_k +\abs{\lambda_n}}}x_k.$$ V principu, lahko te $b$-je pomnožimo s $e^{i\theta}$, ampak nam to vrne rešitev v isti smeri. 
V bistvu lahko uporabimo vsak par pozitivnih in negativnih lastnih vrednosti. Ta postopek lahko vrne toliko rešitev kot je dvakratno število parov lastnih vrednosti matrike $H$ z nasprotnimi predznaki, če so vse lastne vrednosti različne. Konstruirani rešitvi sta neodvisni in še več, ortogonalni, če $\lambda_k =-\lambda_n$. Predpostavimo, da je $b$ realen.
\begin{posledica}\cite{lipkin}
Dobljena izotropna vektorja sta ortogonalna ($b_1 ^T b_2=0$), če in samo če $\lambda_k=-\lambda_n$.
\end{posledica}
\begin{proof}%gleda wi kot negativno, mi pa mamo abs
$$b_1 ^T b_2=(\sqrt{\lambda_k}x_n+\sqrt{\abs{\lambda_n}}x_k)^T (-\sqrt{\lambda_k}x_1+\sqrt{\abs{\lambda_n}}x_k )= -(\lambda_n +\lambda_k).$$
\end{proof}
Ko sta $A$ in $b$ realna smo dokazali naslednji izrek:
\begin{izrek}
Če je $A$ realna in nedefinitna (t.j. ni pozitivno in negativno definitna), potem obstajata najmanj dva neodvisna realna izotropna vektorja.
\end{izrek}
Da bi pokazali, da imamo neskončno število realnih rešitev in, da bi jih nekaj izračunali, moramo vzeti vsaj tri različne lastne vrednosti z različnimi predznaki (ko obstajajo). Predpostavimo, da imamo $\lambda_1 <0<\lambda_2<\lambda_3$ in naj bo $t_1=\abs{c_1}^2$, $t_2=\abs{c_2}^2$. Veljati mora enačba \eqref{eq:en2}
\begin{equation}\label{trije}
\lambda_1 t_1 +\lambda_2 t_2 +\lambda_3 (1- t_1 -t_2)=(\lambda_1 -\lambda_3)t_1 +(\lambda_2 -\lambda_3)t_2 +\lambda_3=0,
\end{equation}
s pogoji: $t_i \ge 0, i=1,2$ in $t_1 +t_2\le1$. Torej imamo $$t_2=\frac{\lambda_3}{\lambda_3 - \lambda_2} -\frac{\lambda_3 -\lambda_1}{\lambda_3 -\lambda_2}t_1,$$
ki definira premico v $(t_1,t_2)$ ravnini. Preveriti moramo, če ta premica seka trikotnik, definiran s pogoji za $t_1,t_2$. Premica seka $t_1$-os pri $\lambda_3 /(\lambda_3 -\lambda_1)$, kar je več kot 1, saj je $\lambda_1 <0$, $t_2$-os pa pri $\lambda_3 /(\lambda_3 - \lambda_2)$, kar je več kot 1. Ta premica ima negativen naklon. Vse dopustne vrednosti za $t_1$ in $t_2$ so dane z daljico v trikotniku. Zato obstaja neskončno število možnih pozitivnih parov $(t_1,t_2)$.
%%SLIKA
Primer $\lambda_1 <\lambda_2<0<\lambda_3$ je podoben zgornjemu, le da premica seka $t_2$-os pod 1. Potem dobimo rešitve $b$ s kombiniranjem pripadajočih treh lastnih vektorjev. Ta problem za iskanje koeficientov je v treh dimenzijah in možne rešitve $t$-ja so v eni dimenziji. Takšna konstrukcija pripelje do naslednjega izreka:
\begin{izrek}
Če je $n>2$ in $A$ je realna in nedefinitna, ima matrika $H$ vsaj tri različne lastne vrednosti z različnimi predznaki. Potem obstaja neskončno število realnih izotropnih vektorjev.
\end{izrek}
%dokaz v citat
Seveda lahko nadaljujemo z večanjem števila lastnih vrednosti. Če uporabimo štiri različne lastne vrednosti z različnimi predznaki, potem moramo na problem gledati v treh dimenzijah. Prostor, kjer je omejitvam zadoščeno, je tetraeder, torej moramo poiskati presek dane ravnine s tem tetraedrom. 
V splošnem, če imamo $k$ različnih lastnih vrednosti z različnimi predznaki, definira naš problem naslednja enačba:
\begin{equation}
\sum_{i=1}^{k-1} (\lambda_i -\lambda_k)t_i +\lambda_k =0, \quad t_i\ge0, i=1, \dots,k-1, \sum_{i=1}^{k-1}t_i \le1.
\end{equation}
Prva enačba opisuje hiperravnino v kateri moramo poiskati  presečišča te hiperravnine z volumnom telesa definiranega s pogoji.
Če je $A$ realna matrika smo končali, saj smo pokazali, da lahko poračunamo toliko realnih rešitev kot hočemo. Posebej lahko sčasoma izračunamo $n$ neodvisnih rešitev, čeprav to ne definira izotropni podprostor. 
%SLED??

%3. poglavje
\section{Kompleksne matrike}
V tem razdelku si bomo pogledali kompleksne matrike. Predstavljeni bodo trije teoretični postopki, kako priti do izotropnih vektorjev, treh avtorjev iz \cite{meurant},\cite{carden} in \cite{trije}.
\subsection{Iskanje izotropnih vektorjev}
\subsubsection{Meurant 1.}
V nekaterih primerih lahko izračunamo rešitve s samo enim ra\-ču\-na\-njem lastnih vrednosti in vektorjev matrike $K$, vendar to ne deluje vedno. Sredstvo, ki lahko pomaga je, da uporabimo lastne vektorje matrike $H$. Če ima matrika $A$ kompleksne elemente, nam prejšnja konstrukcija za realne matrike vrne le vektorje za katere je $\Re(b^\ast Ab)=0$. Najprej opazimo, da lahko v nekaterih primerih uporabimo podobno konstrukcijo kot v prejšnjem razdelku, ki najde množico rešitev za hermitsko matriko $H$ z ničelnim realnim delom in pozitivnim in negativnim imaginarnim delom???. Z uporabo treh lastnih vektorjev $H$, obstaja neskončno rešitev dobljenih na daljici v trikotniku omejitev. Ko točko nenehno sprehajamo po daljici, se tudi imaginarni del rešitve nenehno spreminja. Če sta imaginarna dela, ki ustrezata robnima točkama daljice, različnih predznakov, potem iz izreka o povprečni vrednosti, da obstaja točka na daljici, ki ima ničeln imaginarni del.??To se lahko izračuna z dihotomijo??. Opomba: to vsebuje samo izračun kvadratne forme $x^\ast Ax$. Ne potrebujemo nobenih izračunov lastnih vrednosti in vektorjev. Vendar, se sprememba predznaka v vrednostih $x^\ast Ax$ ne zgodi za nobene trojčke lastnih vrednosti.
\subsubsection{Meurant 2.}
Druga konstrukcija algoritma v \cite{meurant} uporabi lastne vrednosti in lastne vektorje matrike $K=(A-A^\ast)/(2i)$, ki je hermitska. Konstrukcija v 2. razdelku vrne vektorje $b$, tako da je $\Im(b^\ast Ab)=0$. S kombiniranjem lastnih vektorjev matrike $K$ pripadajočim k pozitivnim in negativnim lastnim vrednostim, lahko (v nekaterih primerih) izračunamo dva vektorja $b_1$ in $b_2$, taka da $\alpha_1=\Re(b_1^\ast Ab_1)<0$ in $\alpha_2=\Re(b_2^\ast Ab_2)>0$. 
\begin{lema}\label{komp}
Naj bosta $b_1$ in $b_2$ enotska vektorja z $\Im(b_i^\ast Ab_i)=0$, $i=1,2$ in $\alpha_1=\Re(b_1^\ast Ab_1)<0, \alpha_2=\Re(b_2^\ast Ab_2)>0$. Naj bo $b(t,\theta)=e^{-i\theta}b_1 + tb_2, t,\theta \in \R, \alpha(\theta)=e^{i\theta}b_1^\ast Ab_2 +e^{-i\theta}b_2^\ast Ab_1.$ Potem je 
$$b(t,\theta)^\ast Ab(t,\theta)=\alpha_2 t^2 +\alpha(\theta)t+\alpha_1,$$ 
$\alpha(\theta)\in\R$, ko $\theta=arg(b_2^\ast Ab_1 -b_1^T\bar{A}\bar{b_2}).$ Za $t_1 =(-\alpha(\theta) +\sqrt{\alpha(\theta)^2 -4\alpha_1\alpha_2})/(2\alpha_2)$, imamo 
$$b(t_1, \theta) \not=0,\quad  \frac{b(t_1,\theta)^\ast}{\norm{b(t_1,\theta)}}A\frac{b(t_1,\theta)}{\norm{b(t_1,\theta)}}=0.$$
\end{lema}
Lema \ref{komp} prikazuje kako se izračuna rešitev iz $b_1$ in $b_2$. Če imamo $b_1$ in $b_2$, taka da  $\alpha_1=\Re(b_1^\ast Ab_1)<0$ in $\alpha_2=\Re(b_2^\ast Ab_2)>0$, smo končali.\\
Opomba: Ko je $A$ realna in imamo $b_1=x_1, b_2=x_2$ za lastne vektorje $H$, potem je $\theta=0$ in lema \ref{komp} pove, kako se izračuna en izotropni vektor. Vendar v kompleksnem primeru, ne moremo vedno najti primerne vektorje $b_1$ in $b_2$. Posebej, $y_i$ lastni vektorji $K$, če imajo vrednosti $\Re(y_i^\ast Hy_j)$ enak predznak, konstrukcija ne deluje. Ekstremen primer je Jordanski blok z kompleksno vrednostjo $\alpha$ na diagonali in elementi 1 na naddiagonali. Potem imamo  $\Re(y_i^\ast Hy_j)=0, i\not=j$ in $y_i^\ast Hy_i=-\Re(\alpha)$. Zato so realni deli $b^\ast Ab$ za vse vektorje $b$, ki se lahko konstruirajo, enaki.\\
Ko ni možno izračunati vseh vektorjev $b_1$ in $b_2$ potrebnih za lemo \ref{komp}, izračunamo lastne vektorje $H$ in uporabimo enako tehniko za matriko $iA$. V primeru neuspeha, kombiniramo lastne vektorje $H$ in $K$. Naj bo $x$ (oz. $y$) lastna vrednost $K$ (oz. $H$), upoštevamo vektorje $X_\theta =\cos(\theta)x+\sin(\theta)y,$ $0\le\theta\le\pi$. Ko gre $\theta$ od 0 do $\pi$, $X_\theta ^\ast AX_\theta$ opisuje elipso v zalogi vrednosti. Za dan par lastnih vektorjev $x,y$ iščemo presečišča elipse z realno osjo. Opazimo, da je $A=H+iK$, torej imamo:

\begin{align*}
X_\theta^\ast AX_\theta &= \cos^2(\theta)(x^\ast Hx + ix^\ast Kx)\\ 
&+ \sin^2(\theta)(y^\ast Hy + iy^\ast Ky)\\ 
&+\sin(\theta)\cos(\theta)(x^\ast Hy +y^\ast Hx +i[x^\ast Ky +y^\ast Kx]).
\end{align*}
Naj bo $\alpha=\Im(x^\ast Hx + ix^\ast Kx), \beta=\Im(y^\ast hy +iy^\ast Ky)$ in $\gamma=\Im(x^\ast Hy +y^\ast Hx +i[x^\ast Ky +y^\ast Kx]).$ Ko enačimo imaginarni del $X_\theta ^\ast AX\theta$ z 0, dobimo enačbo:
$$\alpha \cos^2(\theta) +\beta \sin^2(\theta) +\gamma \sin(\theta)\cos(\theta)=0.$$
Predpostavimo, da $\cos(\theta) \not =0$ in delimo, dobimo kvadratno enačbo za $t=\tan(\theta),$
$$\beta t^2 +\gamma t +\alpha =0.$$
Če ima ta enačba realne rešitve, potem dobimo vrednosti $\theta$, ki nam vrnejo take vektorje $X_\theta$, da $\Im(X_\theta ^\ast AX_\theta)=0$. Ko je velikost problema velika, ne uporabimo te konstrukcije za vse pare lastnih vektorjev, saj nas to lahko preveč stane. Uporabimo samo lastne vektorje, ki pripadajo par najmanjšim in največjim lastnim vrednostim.\\
Če tudi ta konstrukcija ne deluje, uporabimo algoritem iz \cite{trije}, ki je opisan malo naprej v nalogi.
\subsubsection{Carden.}
V tem razdelku opišemo idejo za Cardenov algoritem za dano matriko $A \in \C^{n\times n}$ in $\mu \in \C$. Naj bo $\varepsilon >0$ toleranca (npr. $\varepsilon=10^{-16}\norm{A}$ za dvojno natančnost). \\
Najprej izračunamo najbolj levo in najbolj desno lastno vrednost $H_\theta =(e^{i\theta}A+e^{-i\theta}A^\ast)/2$ za $\theta =0,\pi/2$. S tem dobimo zunanjo aproksimacijo zaloge vrednosti, ki jo seka v v robnih točkah $\partial W(A)$. Zunanja aproksimacijo bo pravokotnik, katerega stranice so vzporedne z realno in imaginarno osjo. Če $\mu$ ni v zunanji aproksimaciji, potem $\mu \not \in W(A)$ in ustavimo algoritem, drugače nadaljujemo. Če je višina ali širina zunanje aproksimacije manj kot $\varepsilon$, potem je $W(A)$ približno hermitska ali poševno-hermitska (ali kompleksen premik katere od teh). Če sta višina in širina zunanje aproksimacije manj kot $\varepsilon$, potem je zaloga vrednosti približno točka. V obeh primerih je potrebno ugotoviti ali $\mu \in W(A)$ ter, če je, je potrebno poiskati pripadajoč izotropni vektor. Če $\mu \not \in W(A)$ nadaljujemo z konstrukcijo notranje aproksimacije $W(A)$ z uporabo lastnih vektorjev najbolj leve in desne lastne vrednosti $H_{\theta}$. Notranja aproksimacija bo štirikotnih z oglišči v robnih točkah $\partial W(A)$. Če $\mu$ leži v notranji aproksimaciji, lahko poiščemo izotropni vektor, ki generira $\mu$ (način bo opisan pozneje). Če $\mu$ ne leži v notranji aproksimaciji je potrebno določiti katera stranica notranje aproksimacije mu leži najbližje. Potrebno je izračunati $\hat{\mu}$, ki je najbližja točka do $\mu$ od notranje aproksimacije. Če je $\abs{\hat{\mu}-\mu}<\varepsilon$, lahko izračunamo izotropni vektor za $\hat{\mu}$ in ga sprejmemo kot izotropni vektor za $\mu$ ter ustavimo algoritem. V naslednjem koraku je potrebno posodobiti notranjo in zunanjo aproksimacijo z izračunom največje lastne vrednosti in pripadajočega lastnega vektorja $H_{\theta}$, kjer je smer $\theta$ pravokotna na stranico notranje aproksimacije, ki je najbližja $\mu$. Če ne dobimo nove robne točke, ki se ni dotikala notranje aproksimacije, potem $\mu \not \in W(A)$. Drugače ponovno preverimo, če je $\mu$  v novi zunanji aproksimaciji. Če je, ponovimo isti postopek, drugače $\mu \not \in W(A)$. \\
Ta algoritem ne izkoristi dejstva, da je zaloga vrednosti $2\times2$ matrike elipsa. Ta lastnost nakazuje, da z točkami in pripadajočimi izotropnimi vektorji za notranjo aproksimacijo, lahko natančno določimo izotropne vektorje za točke zunaj notranje aproksimacije. V nadaljevanju bomo algoritem spremenili, da bo upošteval to lasnost.
 %Definiraj genarirajoč vektor in ritzove vrednosti
\subsubsection{Chorianopoulos, Psarrakos in Uhlig.}
V tem razdelku opišemo algoritem Chorianopoulosa, Psarrakosa in Uhliga za inverzen problem zaloge vrednosti. 
Algoritem je hitrejši in daje natančne numerične rezultate tam, kjer se zgornja algoritma mnogokrat ustavita. Tak primer je ko $\mu$, $\mu \in W(A)$ ali $\mu \not\in W(A)$, leži zelo blizu roba zaloge vrednosti $\partial W(A)$. Ta algoritem uporabi za iskanje izotropnih vektorjev le nekaj najbolj osnovnih lastnosti zaloge vrednosti poleg konveksnosti kot je $W(\alpha A +\beta I)=\alpha W(A) +\beta$.\\
Za iskanje izotropnih vektorjev izračunamo do 4 robne točke zaloge vrednost $p_i$ in njihove izotropne vektorje $b_i$ za $i=1,2,3,4$. To storimo z izračunom ekstremnih lastnih vrednosti, ki pripradajo enotskim lastnim vektorjem $x_i$ %a je to isti x oz. b?
matrike $H$ in $K$. Iz $p_i =b^\ast _i Ab_i$ dobimo 4 točke zaloge vrednosti $p_i$, ki so ekstremni horizontalni in vertikalni raztegi zaloge vrednosti $W(A)$. Te točke bomo označili s $rM$ in $rm$ za maksimalen in minimalen horizontalni razteg $W(A)$ in s $iM$ in $im$ za maksimalen in minimalen vertikalen razteg $W(A)$. Če katera od teh točk leži absolutno v bližini $10^{-13}$  točke 0, potem je naš izotropni vektor kar pripadajoč enotski vektor. Če pa med izračunom lastnih vrednosti in vektorjev ugotovimo, da je ena od hermitskih matrik $H$ in $iK$ definitna, potem vemo da $\mu \not\in W(A)$ in algoritem ustavimo.\\
%%%WTF JE TA ODSTAVEK???? 
%Véliki króg (tudi ortodromíja) je krožnica na površini krogle, oziroma na sferi, ki ima enak obseg kot krogla ali sfera.
V naslednjem koraku moramo poiskati presečišča realne osi z elipsami velikega kroga, ki grejo skozi vse naše možne pare (kjer imajo imaginarni deli nasprotne predzanke) robnih točk $p_i=x^\ast Ax, p_j=y^\ast  Ay$. Če dobimo presečišča na obeh straneh 0, potem izračunamo generirajoč enotski vektor za $0\in W(A)$ z uporabo leme \ref{komp}.
V nasprotnem primeru moramo poračunati kvadratno enačbo, katere ničle določijo koordinatne osi $W(A)$ točk na elipsah skozi točke $x^\ast Ax,y^\ast Ay\in \partial W(A)$ %?????
ki se generirajo s točkami v $\C^n$ na velikem krogu skozi $x$ in $y$.
\begin{multline}\label{eq:en3}
(tx +(1-t)y)^\ast A(tx+(1-t)y)=(x^\ast Ax+y^\ast Ay -(x^\ast Ay +y^\ast Ax))t^2 \\
+(-2y^\ast Ay +(x^\ast Ay+y^\ast Ax))t +y^\ast Ay.
\end{multline}
To je kvadratna enačba z kompleksnimi števili. Nas zanimajo samo rešitve, ki imajo imaginaren del enak 0, saj želimo uporabiti lemo \ref{komp}.
Če imaginarni del enačbe \eqref{eq:en3} enačimo z 0, dobimo naslednji polinom z realnimi koeficienti:
\begin{equation}
t^2+gt+\frac{p}{f}=0 \label{eq:en4}
\end{equation}
za $q=\Im(x^\ast Ax)$, $p=\Im(y^\ast Ay)$ in $r=\Im(x^\ast  Ay + y^\ast Ax)$. Označimo $f=p+q-r$ in $g=(r-2p)/f$. Enačba \eqref{eq:en4} ima dve realni rešitvi $t_i$, $i=1,2$, ki vrneta dva generirajoča vektorja $b_i=t_ix+(1-t_i)y$ ($i=1,2$), za dve realni točki. Z normalizacijo dobimo željene entoske izotropne vektorje. Če nobena od možnih elips ne vrne dve realni točki zaloge vrednosti na vsaki strani 0, potem preverimo, če to stori njihova skupna množica. Če ne, potem izračunamo še več lastnih vrednosti in lastnih vektorjev za $A(\theta)=\cos(\theta)H+\sin(\theta)iK$, za kote $\theta$ drugačne od $\theta \not =0,\pi/2$. To delamo dokler dobljene elipse generirane z velikim krogom znotraj $W(A)$ ne sekajo realne osi na obeh straneh 0 in lahko izračunamo izotropne vektorje z lemo \ref{komp} ali dokler ena od matrik $A(\theta)$ ne postane definitna.
%SLIKA
\section{Algoritmi/Numerična analiza}
\section{Zaključek}



\vfill

% seznam uporabljene literature
% \cite{meurant} za referenco
\begin{thebibliography}{99}

\bibitem{meurant}
G. Meurant, \emph{The computation of isotropic vectors}, Numer. Alg. {\bf 60} (2012) 193--204.

\bibitem{carden}
R. Carden, \emph{A simple algorithm for the inverse field of values problem}, Inverse Probl. {\bf 25} (2009) 1--9

\bibitem{trije}
C. Chorianopoulos, P. Psarrakos in F. Uhlig, \emph{A method for the inverse numerical range problem}, Electron. J. Linear Algebra {\bf 20} (2010) 198--206

\bibitem{lipkin}
N. Ciblak, H. Lipkin, \emph{Orthonormal isotropic vector bases}, In: Proceedings of DETC'98, 1998 ASME Design Engineering Technical Conferences (1998).

\bibitem{num}
Johnson, C. R., \emph{Numerical determination of the field of values of a general complex matrix}, SIAM J. Numer. Anal. {\bf15} (1978) 595--602.


\end{thebibliography}

\end{document}

