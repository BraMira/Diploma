\documentclass{beamer}



\mode<presentation>
{
  \usetheme{Frankfurt}
  % or ...

  %\setbeamercovered{transparent}
  % or whatever (possibly just delete it)
}
\usepackage[slovene]{babel}
\usepackage[utf8]{inputenc}
%\usepackage[pdftex]{graphicx}
%\usepackage{color}
%\usepackage{float}
%\usepackage{eurosym}

\usepackage{amsfonts}
\usepackage{amsmath,amsthm}
\usepackage{url}
\usepackage{tikz}
\usepackage{lipsum}

%\usepackage[english]{babel}
% or whatever

%\usepackage[latin1]{inputenc}
% or whatever

%\usepackage{beamerthemeshadow}
\usepackage{times}
%\usepackage[T1]{fontenc}
% Or whatever. Note that the encoding and the font should match. If T1
% does not look nice, try deleting the line with the fontenc.

\newcommand{\bx}{\mathbf{x}}
\DeclareMathOperator{\Gl}{Gl}
\DeclareMathOperator{\adj}{adj}
%\newcommand{\Gl}{{\mathrm {GL}}\,}
%\newizrek{proposition}[izrek]{Proposition}

% finan�ni instrumenti
\newcommand{\PV}{\mathrm{PV}}


\newcommand{\ds}{\displaystyle}
\newcommand{\ts}{\textstyle}
\newcommand{\presledek}{\vspace{3mm}}
\newcommand{\ph}{\phantom{1}} % pomoč pri poravnavi teksta v slikah TikZ


\newcommand{\abs}[1]{ \left\lvert#1\right\rvert} 
\newcommand{\norm}[1]{\left\lVert#1\right\rVert}
\newcommand{\Co}{\operatorname{Co}} %konveksna ogrinjača

\newcommand{\R}{\mathbb R}
\newcommand{\N}{\mathbb N}
\newcommand{\Z}{\mathbb Z}
\newcommand{\C}{\mathbb C}
\newcommand{\Q}{\mathbb Q}
\newtheorem{izrek}{Izrek}
\newtheorem{lema}[izrek]{Lema}
\newtheorem{trditev}[izrek]{Trditev}
\newtheorem{posledica}[izrek]{Posledica}
\newtheorem{definicija}[izrek]{Definicija}
\newtheorem{naloga}[izrek]{Naloga}
\newtheorem{resitev}[izrek]{Naloga}
\setbeamertemplate{lema}[numbered]
\newcounter{saveenumi}
\newcommand{\seti}{\setcounter{saveenumi}{\value{enumi}}}
\newcommand{\conti}{\setcounter{enumi}{\value{saveenumi}}}
\resetcounteronoverlays{saveenumi}
\setbeamertemplate{bibliography item}{\insertbiblabel}

\title[Računanje izotropnih vektorjev] % (optional, use only with long paper titles)
{Računanje izotropnih vektorjev}

%\subtitle
%{Presentation Subtitle} % (optional)

\author[Mirjam Pergar] % (optional, use only with lots of authors)
{\textbf{Avtor:}  Mirjam Pergar\\
\textbf{Mentor:} prof. dr. Bor Plestenjak
}
% - Use the \inst{?} command only if the authors have different
%   affiliation.

\institute[Fakuleta za matematiko in fiziko] % (optional, but mostly needed)
% \inst{FMF}
%

%  \and
%  \inst{2}%
%  Department of Theoretical Philosophy\\
%  University of Elsewhere

% - Use the \inst command only if there are several affiliations.
% - Keep it simple, no one is interested in your street address.

\date[SEPTEMBERRRR] % (optional)
{SEPTEMBEER}

% Delete this, if you do not want the table of contents to pop up at
% the beginning of each subsection:
%\AtBeginSubsection[]
%{
%  \begin{frame}<beamer>
%    \frametitle{Outline}
%    \tableofcontents[currentsection,currentsubsection]
%  \end{frame}
%}


% If you wish to uncover everything in a step-wise fashion, uncomment
% the following command:

%\beamerdefaultoverlayspecification{<+->}
\setbeamertemplate{footline}[frame number]

\begin{document}

\begin{frame}
  \titlepage
\end{frame}

\begin{frame}
  \frametitle{Vsebina}
  \tableofcontents
  % You might wish to add the option [pausesections]
\end{frame}


\section{Uvod}
\subsection{Problem}
\begin{frame}
  \frametitle{Problem}
\begin{alertblock}{}
Naj bo $A\in\R^{n\times n} (\C^{n\times n})$, $det(A)\ne 0$. Iščemo enotski vektor $b$, da je
\begin{equation}\label{eq:zac}
b^\ast Ab=0
\end{equation}
pravimo mu \textbf{izotropni vektor}. \medskip \\
Bolj splošen je inverzni problem numeričnega zaklada, kjer iščemo enotski vektor $b$, za katerega velja:
\begin{equation}\label{eq:splosno}
b^\ast Ab=\mu,
\end{equation}
kjer je $\mu \in \C$ dano število.
\end{alertblock}
\end{frame}
\begin{frame}
\begin{itemize}
\item Problem \eqref{eq:splosno} prevedemo na problem \eqref{eq:zac} za drugo matriko $ (A-\mu I)$.\medskip
\item Če $\mu$ lastna vrednost matrike $A$, potem je rešitev pripadajoč lastni vektor matrike $A$.\medskip
\item Če $\mu$ ni lastna vrednost matrike $A$, je $det(A-\mu I)\ne 0$ in moramo lastne in izotropne vektorje izračunati.\medskip
\item Od sedaj  $\mu =0$.\medskip
\item Hermitski del matrike $A$ bomo označili s $H=(A+A^\ast)/2$.\medskip
\item Poševno-hermitski del matrike $A$ bomo označili z \medskip$K=(A-A^\ast)/2\imath$.
\end{itemize}
\end{frame}

\subsection{Numerični zaklad}
\begin{frame}
\frametitle{Numerični zaklad}
\begin{definicija}
\textbf{Numerični zaklad} matrike $A \in \C^{n\times n}$ je podmnožica kompleksne ravnine, definirana kot
$$W(A)=\{x^\ast Ax: x \in \C^n, x^\ast x=1\}.$$
\end{definicija}\pause
\begin{itemize}
\item $W(A)$  je množica vseh Rayleighovih kvocientov matrike $A$.
\item Razberemo lahko informacije o matriki in pogosto da več informacij kot spekter.
%\item Lastne vrednosti hermitskih in normalnih matrik imajo uporabne lastnosti, s katerimi si pomagamo pri njegovem izračunu.
\item $W(A)$ se preslika na realno os in ostane le daljica, če vzamemo hermitski del matrike.
\item Veljati mora $0 \in W(A)$, če hočemo, da ima \eqref{eq:zac} vsaj eno rešitev.
\end{itemize}

%SLIIIIIIKEEEE

\end{frame}

\begin{frame}
\frametitle{Lastnosti numeričnega zaklada}
\begin{enumerate}[(i)]
\item $W(A)$ je konveksna in kompaktna podmnožica $\C$.
\item $\sigma(A)\subseteq W(A)$, kjer $\sigma(A)$ označuje spekter.
\item Za vsako unitarno matriko $U$ je $W(U^\ast AU)=W(A).$

\item Seštevanje s skalarjem: $W(A+zI)=W(A)+z$ za $\forall z \in \C$.

\item Množenje s skalarjem: $W(zA)=zW(A)$ za $\forall z \in \C$.

\item Subaditivnost: Za $\forall A, B \in \C^{n\times n}$ velja $W(A+B) \subseteq W(A) +W(B).$

\item Projekcija: $W(H)= \Re( W(A))$, za $\forall A\in \C^{n\times n}$ kjer s $H$ označimo hermitski del matrike $A$.
\item Če je $A$ normalna, potem $W(A)=\Co(\sigma(A))$, kjer s $\Co$ označimo zaprto konveksno ogrinjačo množice.

\item $W(A)$ je daljica na realni osi, če in samo če, je $A$ hermitska.

\end{enumerate}
\end{frame}

\subsection{Uporaba}
\begin{frame}
\frametitle{Uporaba}
\begin{itemize}
\item Preučevanje konvergence nekaterih iterativnih metod za reševanje linearnih sistemov, npr. GMRES.\medskip
\item Aplikacije v numerični analizi, diferencialnih enačbah, teoriji sistemov itd.
\end{itemize}
\end{frame}
\section{Realne matrike} %REALNE MATRIKE
\subsection{Uvod}
\begin{frame}
\frametitle{Realne matrike}
\begin{block}{}
Ko je $A\in\R^{n\times n}$ nas zanima, kako se izračuna rešitev enačbe:
\begin{equation}\label{eq:realna}
b^\ast Hb=0,
\end{equation}
kjer je $H\in\R^{n\times n}$ simetrična matrika (t.j. $H=H^T$).
\end{block}\pause
\begin{lema} %\cite{lipkin}
Izotropni vektorji matrike A so identični izotropnim vektorjem njenega simetričnega dela.
\end{lema} \pause
%Velja enakost:
%\begin{block}{}
%$$b^T Ab=0 \Leftrightarrow b^T (A+A^T)b=0.$$ 
%\end{block}\pause
Vemo:
\begin{itemize}
\item $W(A)$ je simetričen glede na realno os.
\item $0 \in W(A)$, če in samo če $\lambda_n\le0\le\lambda_1$, kjer sta $\lambda_n$ in $\lambda_1$ najmanjša in največja lastna vrednost matrike $H$.
\end{itemize}
\end{frame}
\begin{frame}
Naj bosta $x_1, x_n \in \R^n$ lastna vektorja, pripadajoča $\lambda_1$ in $\lambda_n$. Potem sta:
\begin{itemize}
\item $x_1^T Ax_1=x_1^T Hx_1=\lambda_1$,
\item $x_n^T Ax_n=x_n^T Hx_n=\lambda_n$
\end{itemize}
realni točki na skrajni levi in skrajni desni $W(A)$ na realni osi.
\begin{itemize}\pause
\item Realne rešitve \eqref{eq:realna} izračunamo z uporabo lastnih vektorjev matrike $H$. 
\item Predpostavimo, da iščemo vektorje $b$ z normo 1.
\item $H$ zapišemo kot $$H=X\Lambda X^T,$$ kjer je $\Lambda=\lambda_i I$ in $X$ je ortogonalna matrika lastnih vektorjev, tako da $X^T X=I$.
\end{itemize}
\end{frame}
\begin{frame}
\begin{itemize}
\item Uporabimo ta spektralni razcep v \eqref{eq:realna}: $$b^\ast Hb=b^\ast X\Lambda X^T b=0.$$ 
\item  Označimo s $c=X^Tb$ vektor projekcije $b$ na lastne vektorje matrike $H$. 
\end{itemize}\pause
\begin{izrek} \label{izrek2}
Naj bo $b$ rešitev problema \eqref{eq:realna}. Potem vektor $c=X^T b$ s komponentami $c_i$ zadošča naslednjima enačbama:
\begin{align}
\sum_{i=1}^{n} \lambda_i \abs{c_i}^2=0 \label{eq:en1},\\
\sum_{i=1}^{n}\abs{c_i}^2=1. \label{eq:en2}
\end{align}
\end{izrek}
\end{frame}

\subsection{Iskanje izotropnih vektorjev}
\begin{frame}
\frametitle{Iskanje izotropnih vektorjev}
\begin{block}{}
Če niso vse lastne vrednosti H enako predznačene, potem mora za najmanjšo veljati $\lambda_n <0$. \medskip \\
 Naj bo $k<n$ tak, da je $\lambda_k >0$ in $0<t<1, t\in \R$ . Izberemo \medskip taka $c_n$ in $c_k$, da velja  $\abs{c_n}^2 =t$, $\abs{c_k}^2=1-t$ in $c_i =0$, $i\not=n,k$, \medskip  ker velja enačba $\sum_{i=1}^{n}\abs{c_i}^2=1$ ($t+ (1-t)=1$). Iz \medskip $\sum_{i=1}^{n} \lambda_i \abs{c_i}^2=0$ mora veljati enačba: $$\lambda_n t +\lambda_k (1-t)=0,$$ katere rešitev je:
\begin{equation*}
t_s=\frac{\lambda_k}{\lambda_k -\lambda_n}.
\end{equation*}
\end{block}
\end{frame}
\begin{frame}
\begin{itemize}
\item Absolutna vrednost $c_n$ (oz. $c_k$) je kvadratni koren od $t_s$ (oz. $1-t_s$).
\item Ker je $b=Xc$, sta realni rešitvi: $$b_1=\sqrt{t_s}x_n +\sqrt{1-t_s}x_k,\quad b_2=-\sqrt{t_s}x_n+\sqrt{1-t_s}x_k,$$ kjer sta $x_n$ in $x_k$ lastna vektorja pripadajoča  $\lambda_n$ in $\lambda_k$.\pause
\item Ker imata izraza v rešitvah enaka imenovalca, lahko rešitvi zapišemo kot: $$b_1=\sqrt{\lambda_k}x_n+\sqrt{\abs{\lambda_n}}x_k, \quad b_2=-\sqrt{\lambda_k}x_n+\sqrt{\abs{\lambda_n}}x_k.$$ \pause
%(sledi iz \cite{lipkin}).
\item Vektor mora biti normiran:
\begin{align*}
b_1=\sqrt{\frac{\lambda_k}{\lambda_k +\abs{\lambda_n}}}x_n + \sqrt{\frac{\abs{\lambda_n}}{\lambda_k +\abs{\lambda_n}}}x_k,\\ b_2=-\sqrt{\frac{\lambda_k}{\lambda_k +\abs{\lambda_n}}}x_n + \sqrt{\frac{\abs{\lambda_n}}{\lambda_k +\abs{\lambda_n}}}x_k.
\end{align*}
\end{itemize}
% Konstruirani rešitvi sta neodvisni in še več, ortogonalni, če $\lambda_k =-\lambda_n$. 
\end{frame}

\begin{frame}
\frametitle{Neskončno rešitev}
Ko sta $A\in\R^{n\times n}$ in $b\in\R^n$ smo dokazali naslednji izrek: \medskip
\begin{izrek}
Če je $A\in\R^{n\times n}$ nedefinitna (t.j. ni pozitivno in negativno definitna), potem obstajata najmanj dva neodvisna realna izotropna vektorja.
\end{izrek}\pause \medskip

%Pokazati želimo, da imamo neskončno število realnih rešitev in jih izračunati.\\
Vzeti moramo vsaj tri različne lastne vrednosti, ki ne smejo biti istega predznaka (ko obstajajo).

\end{frame}
\begin{frame}
\begin{itemize}
\item Predpostavimo, da $\lambda_1 <0<\lambda_2<\lambda_3$.\pause
\item Naj bo $t_1=\abs{c_1}^2$, $t_2=\abs{c_2}^2$.\pause
\end{itemize}
Veljati mora enačba \eqref{eq:en2}
\begin{block}{}
\begin{equation*}%\label{trije}
\lambda_1 t_1 +\lambda_2 t_2 +\lambda_3 (1- t_1 -t_2)=0
\end{equation*} oz.
\begin{equation*}
(\lambda_1 -\lambda_3)t_1 +(\lambda_2 -\lambda_3)t_2 +\lambda_3=0,
\end{equation*}
s pogoji: $t_i \ge 0, i=1,2$ in $t_1 +t_2\le1$.
\end{block}{}
\end{frame}
\begin{frame}
\frametitle{Neskončno rešitev}
\begin{itemize}
\item Dobimo premico definirano v $(t_1,t_2)$ ravnini z enačbo:
\begin{block}{}
$$t_2=\frac{\lambda_3}{\lambda_3 - \lambda_2} -\frac{\lambda_3 -\lambda_1}{\lambda_3 -\lambda_2}t_1.$$
\end{block}\pause
\item Pogoji za $t_1,t_2$ definirajo trikotnik.
\item Preverimo, če premica seka trikotnik.
\item Vse dopustne vrednosti za $t_1$ in $t_2$ so dane z daljico v trikotniku. 
\item Zato obstaja neskončno število možnih pozitivnih parov $(t_1,t_2)$.
\end{itemize}
\end{frame}

\subsection{Primer}
\begin{frame}
\frametitle{Zgled}
Poglejmo si enostaven zgled za matriko $3x3$ z lastnimi \medskip vrednostmi $\lambda_1=-1, \lambda_2=1$ in $\lambda_3=2$.\medskip \\ Matrika lastnih vektorjev $X$ je enaka $I$.\medskip \\

Enačba premice je $t_2=2-3t_1$ s pogoji $t_1, t_2 \ge 0$ in $t_1+t_2\le 1$.

\end{frame}
\begin{frame}
\frametitle{Zgled}
Dopustne rešitve za $t_1$ in $t_2$ so dane z daljico v trikotniku.\pause
\begin{center}
\includegraphics[width=7cm]{graf3.jpg}
\end{center}
\end{frame}
\begin{frame}
\frametitle{Zgled}
Iz daljice lahko izberemo katerikoli par točk $(t_1, t_2)$, npr. $(0.5, 0.5)$. Potem vemo kako izgleda vektor 
$c=\begin{bmatrix}
\sqrt{0.5}\\
\sqrt{0.5}\\
0
\end{bmatrix}$. Iz enačbe $c=X^T b$ dobimo 
$$b=Xc=c =\begin{bmatrix}
\sqrt{0.5}\\
\sqrt{0.5}\\
0
\end{bmatrix}.$$
Seveda je rešitev tudi $b=-c$. Tako dobimo neskončno izotropnih vektorjev $b$.
\end{frame}
\begin{frame}
\frametitle{Neskončno rešitev}
%Ta problem za iskanje koeficientov je v treh dimenzijah in možne rešitve $t$-ja so v eni dimenziji. 
Takšna konstrukcija pripelje do naslednjega izreka:
\begin{izrek}
Če je $n>2$ in je $A\in\R^{n\times n}$ nedefinitna, ima matrika $H$ vsaj tri različne lastne vrednosti z različnimi predznaki. Potem obstaja neskončno število realnih izotropnih vektorjev.
\end{izrek}\pause
\begin{block}{Splošen zapis}
\begin{equation*}
\sum_{i=1}^{k-1} (\lambda_i -\lambda_k)t_i +\lambda_k =0, \quad t_i\ge0,  i=1, \dots,k-1, \quad \sum_{i=1}^{k-1}t_i \le1.
\end{equation*}
\end{block}
\end{frame}
\section{Kompleksne matrike} %KOMPLEKSNE MATRIKE
\subsection{Uvod}
\begin{frame}
\frametitle{Kompleksne matrike}
Predstavljeni bodo algoritmi naslednjih avtorjev:\medskip
\begin{enumerate}[1.]
\item \emph{Meurant},
\item \emph{Carden},
\item \emph{Chorianopoulos, Psarrakos in Uhlig} (CPU).
\end{enumerate}\medskip
Algoritme bomo numerično primerjali, saj bi radi, da algoritem vrne izotropni vektor s čim manj računanja.
\end{frame}

\subsection{Meurant}
\begin{frame}
\frametitle{Meurant}
\begin{itemize}
\item Uporabimo lastne vrednosti in lastne vektorje hermitske matrike $K=(A-A^\ast)/(2i)$.
\item S kombiniranjem lastnih vektorjev matrike $K$ pripadajočim k pozitivnim in negativnim lastnim vrednostim, lahko (v nekaterih primerih) izračunamo taka vektorja $b_1$ in $b_2$, da $$\alpha_1=\Re(b_1^\ast Ab_1)<0$$ in $$\alpha_2=\Re(b_2^\ast Ab_2)>0.$$
\end{itemize}
\end{frame}
\begin{frame}
\frametitle{Meurant}
\begin{lema}\label{komp}
Naj bosta $b_1$ in $b_2$ enotska vektorja z $\Im(b_i^\ast Ab_i)=0$, $i=1,2$ in \medskip $\alpha_1=\Re(b_1^\ast Ab_1)<0$,  $\alpha_2=\Re(b_2^\ast Ab_2)>0$. Naj bo $b(t,\theta)=e^{-i\theta}b_1 + tb_2$, $t, \theta \in \R$ in  $\alpha(\theta)=e^{i\theta}b_1^\ast Ab_2 +e^{-i\theta}b_2^\ast Ab_1.$ Potem je 
$$b(t,\theta)^\ast Ab(t,\theta)=\alpha_2 t^2 +\alpha(\theta)t+\alpha_1,$$ 
$\alpha(\theta)\in\R$, ko $\theta=arg(b_2^\ast Ab_1 -b_1^T\bar{A}\bar{b_2}).$\medskip \\ Za $t_1 =(-\alpha(\theta) +\sqrt{\alpha(\theta)^2 -4\alpha_1\alpha_2})/(2\alpha_2)$  imamo 
$$b(t_1, \theta) \not=0,\quad  \frac{b(t_1,\theta)^\ast}{\norm{b(t_1,\theta)}}A\frac{b(t_1,\theta)}{\norm{b(t_1,\theta)}}=0.$$
\end{lema}
\end{frame}

\begin{frame}
\frametitle{Algoritem}
\begin{enumerate}[1.]
\item S kombiniranjem lastnih vektorjev $K$, pripadajočim pozitivnim in negativinim lastnim vrednostim, izračunamo taka $b_1$ in $b_2$, da  $\alpha_1=\Re(b_1^\ast Ab_1)<0$ in $\alpha_2=\Re(b_2^\ast Ab_2)>0$. Uporabimo lemo \ref{komp} in končamo.\medskip
\item Če ne najdemo $b_1$, $b_2$ potrebna za lemo \ref{komp}, izračunamo še lastne vektorje matrike $H$.  Ponovimo korak 1. za matriko $\imath A$.\medskip
\item Če postopek ne deluje niti za $\imath A$, uporabimo kombinacijo lastnih vektorjev $K$ in $H$, kjer z $x$ označimo lastni vektor $K$ in z $y$ lastni vektor $H$.\medskip
\item Upoštevamo vektorje $X_\theta =\cos(\theta)x+\sin(\theta)y$$, 0\le\theta\le\pi$. $X_\theta ^\ast AX_\theta$ opiše elipso znotraj $W(A)$.
\seti
\end{enumerate}
\end{frame}
\begin{frame}
\frametitle{Algoritem}
\begin{enumerate}[1.]
\conti
\item Za dan par $(x,y)$ iščemo presečišča elipse $X_\theta ^\ast AX_\theta$ z realno osjo. U\-po\-šte\-va\-mo, da je $A=H+iK$: 
\begin{align*}
X_\theta^\ast AX_\theta &= \cos^2(\theta)(x^\ast Hx + ix^\ast Kx) + \sin^2(\theta)(y^\ast Hy + iy^\ast Ky)\\ 
&+\sin(\theta)\cos(\theta)(x^\ast Hy +y^\ast Hx +i[x^\ast Ky +y^\ast Kx]).
\end{align*}
Naj bo $\alpha=\Im(x^\ast Hx + ix^\ast Kx)$, $\beta=\Im(y^\ast hy +iy^\ast Ky)$ in\medskip $\gamma=\Im(x^\ast Hy +y^\ast Hx +i[x^\ast Ky +y^\ast Kx]).$ Ko izenačimo $\Im(X_\theta ^\ast AX_\theta)=0$, dobimo enačbo:
$$\alpha \cos^2(\theta) +\beta \sin^2(\theta) +\gamma \sin(\theta)\cos(\theta)=0.$$
Predpostavimo, da $\cos(\theta) \not =0$ in delimo, dobimo kvadratno enačbo za $t=\tan(\theta),$
$$\beta t^2 +\gamma t +\alpha =0.$$
\seti
\end{enumerate}
\end{frame}
\begin{frame}
\frametitle{Algoritem}
\begin{enumerate}[1.]
\conti
\item Če ima ta enačba realne rešitve, potem dobimo vrednosti $\theta$, ki nam vrnejo take vektorje $X_\theta$, da $\Im(X_\theta ^\ast AX_\theta)=0$.\medskip
\item Če tudi ta konstrukcija ne deluje, uporabimo algoritem CPU.
\end{enumerate}
\pause
Razlike v implementaciji v Matlabu:
\begin{itemize}
\item Izračunamo 3 največje in 3 najmanjše lastne vrednosti in lastne vektorje.\pause
\item $b_1$ in $b_2$ najdemo tako, da iščemo presečišča realne osi z maksimizirano elipso $X_{\theta}^\ast A X_{\theta}$.\pause
\item Elipso maksimiziramo tako, da vzamemo $X_{\theta}(\phi) = \cos (\theta) x \exp ^{\phi \imath} + \sin (\theta)y$, kjer je $\phi = \frac{1}{2\imath}\textnormal{ln}(x^\ast Ay(y^\ast Ax)^{-1})$.
\end{itemize}


%SLIKA KAKO DELUJE

%OPIS AlgM kakšne so razlike
\end{frame}

\subsection{Carden}
\begin{frame}
\frametitle{Carden}

\begin{itemize}
\item Predpostavimo, da je $\mu$ v konveksni ogrinjači treh točk $\theta_i \in W(A)$, za katere smo lahko izračunali izotropne vektorje $b_i$. 
\item Konveksna ogrinjača $\theta_i$ je trikotnik (lahko je izrojen).
\item Radi bi, da je $\mu$ na daljici, ki ima take robne točke, da za njih vemo ali lahko izračuamo izotropne vektorje. BSŠ predpostavimo, da je $\theta_1$ ena od robnih točk te daljice. Za drugo robno točko vzamemo $w$, ki je presečišče daljice med $\theta_2$ in $\theta_3$ s premico, ki teče skozi $\theta_1$ in $\mu$.
\end{itemize}
\end{frame}
\begin{frame}
\frametitle{Carden}
\begin{itemize}
\item $w$ je konveksna kombinacija $\theta_2$ in $\theta_3$, zato mu lahko določimo pripadajoč izotropni vektor. Ker pa je $\mu$ konveksna kombinacija $w$ in $\theta_1$, lahko tudi njemu določimo izotropni vektor.
\end{itemize}
\begin{center}
\includegraphics[width=5.5cm]{triangle.jpg}
\end{center}
\end{frame}
\begin{frame}
\frametitle{Algoritem}
Naj bo $\varepsilon >0$ (npr. $\varepsilon=10^{-16}\norm{A}$). 
\begin{enumerate}[1.]
\item Poiščemo zunanjo aproksimacijo $W(A)$, z izračunom najbolj leve in desne lastne vrednosti $H_\theta =(e^{i\theta}A+e^{-i\theta}A^\ast)/2$ za $\theta =0, \pi/2$. %S tem dobimo zunanjo aproksimacijo zaloge vrednosti, ki seka $\partial W(A)$ v robnih točkah. %RAZLAGA Zunanja aproksimacija bo pravokotnik, katerega stranice so vzporedne z realno in imaginarno osjo. 
Če $\mu$ ni v zunanji aproksimaciji, potem $\mu \not \in W(A)$ in ustavimo algoritem, drugače nadaljujemo.\medskip
\item Če je višina ali širina zunanje aproksimacije manj kot $\varepsilon$, potem je $W(A)$ približno hermitska ali poševno-hermitska. Če sta višina in širina zunanje aproksimacije manj kot $\varepsilon$, potem je $W(A)$ približno točka. V obeh primerih lahko ugotovimo ali je $\mu \in W(A)$. Če je, poiščemo pripadajoč izotropni vektor, drugače nadaljujemo.
\item Nadaljujemo s konstrukcijo notranje aproksimacije $W(A)$ z uporabo lastnih vektorjev najbolj leve in desne lastne vrednosti $H_{\theta}$.% RAZLAGA Notranja aproksimacija bo štirikotnik z oglišči v robnih točkah $\partial W(A)$.

\seti
\end{enumerate}
\end{frame}
\begin{frame}
\frametitle{Algoritem}
\begin{enumerate}[1.]
\conti
\item  Če $\mu$ leži v notranji aproksimaciji, lahko poiščemo izotropni vektor.Uporabimo postopek opisan na začetku. %OPISS
Če $\mu$ ne leži v notranji aproksimaciji, določimo katera stranica notranje aproksimacije mu leži najbližje.\medskip
\item  Izračunamo $\hat{\mu}$, ki je najbližja točka do $\mu$, ki leži na notranji aproksimaciji. Če je $\abs{\hat{\mu}-\mu}<\varepsilon$, izračunamo izotropni vektor za $\hat{\mu}$ in ga sprejmemo kot izotropni vektor za $\mu$ ter ustavimo algoritem.
\seti
\end{enumerate}
\end{frame}
\begin{frame}
\frametitle{Algoritem}
\begin{enumerate}[1.]
\conti
\item Posodobimo notranjo in zunanjo aproksimacijo z izračunom največje lastne vrednosti in pripadajočega lastnega vektorja $H_{\theta}$, kjer je smer $\theta$ pravokotna na stranico notranje aproksimacije, ki je najbližja $\mu$. Če ne dobimo nove robne točke, ki se ni dotikala notranje aproksimacije, potem $\mu \not \in W(A)$. \medskip
\item Preverimo, če je $\mu$  v novi zunanji aproksimaciji. Če je, se vrnemo na 4. korak, drugače $\mu \not \in W(A)$. 
\end{enumerate}
\end{frame}


\subsection{Chorianopoulos, Psarrakos in Uhlig}
\begin{frame}
\frametitle{Chorianopoulos, Psarrakos in Uhlig - CPU}

\begin{enumerate}[1.]
\item Izračunamo do 4 robne točke $W(A)$, $p_i$ in njihove izotropne vektorje $b_i$ za $i=1,2,3,4$, tako da izračunamo ekstremne lastne vrednosti, ki pripadajo enotskim lastnim vektorjem $x_i$ matrike $H$ in $K$.\medskip
\item Nastavimo $p_i =b^\ast _i Ab_i$. Dobimo štiri točke $p_i$, ki označujejo ekstremne vrednosti $W(A)$, tj. najmanjši in največji horizontalni in vetikalni razteg. Označimo jih z $rM$ in $rm$ za maksimalen in minimalen horizontalni razteg $W(A)$ in z $iM$ in $im$ za maksimalen in minimalen vertikalen razteg $W(A)$. Če je $|p_i|<10^{-13}$ za $i=1,2,3,4$, potem je naš izotropni vektor kar pripadajoč enotski vektor.\medskip
\item Če med računanjem lastnih vektorjev in lastnih vrednosti ugotovimo, da je ena izmed matrik $H$ in $K$ definitna, t.j. da imajo njene lastne vrednosti vse enak predznak, potem vemo, da $\mu \not\in W(A),$ in algoritem ustavimo.
\seti
\end{enumerate}
\end{frame}\begin{frame}
\frametitle{Algoritem}
\begin{enumerate}[1.]
\conti
\item Narišemo elipse, ki so preslikave velikega kroga kompleksne sfere $\C^n$, ki gredo skozi vse možne pare točk $rm, rM, im$ in $iM$, ki imajo nasprotno predznačene imaginarne dele. Nato izračunamo presečišča vsake dobljene elipse z realno osjo.\medskip.
\item Če so izračunana presečišča na obeh straneh 0, potem izračunamo izotropni vektor z lemo \ref{komp}.\medskip
\item Če presečišča niso na obeh straneh 0, potem moramo rešiti kvadratno enačbo
$$(tx +(1-t)y)^\ast A(tx+(1-t)y)  =$$
$$(x^\ast Ax+y^\ast Ay -(x^\ast Ay +y^\ast Ax))t^2 +$$
$$+(-2y^\ast Ay +(x^\ast Ay+y^\ast Ax))t +y^\ast Ay.$$

\seti
\end{enumerate}
\end{frame}

\begin{frame}
\frametitle{Algoritem}
\begin{enumerate}[1.]
\conti
\item Zanimajo nas samo rešitve, ki imajo imaginaren del enak 0, saj želimo uporabiti lemo \ref{komp}.
Če imaginarni del enačbe enačimo z 0, dobimo naslednjo polinomsko enačbo z realnimi koeficienti:
\begin{equation*}
t^2+gt+\frac{p}{f}=0 
\end{equation*}
za $q=\Im(x^\ast Ax)$, $p=\Im(y^\ast Ay)$ in $r=\Im(x^\ast  Ay + y^\ast Ax)$. Označimo $f=p+q-r$ in $g=(r-2p)/f$.\medskip
\item  Enačba ima realni rešitvi $t_i$, $i=1,2$, ki vrneta generirajoča vektorja $b_i=t_ix+(1-t_i)y$ ($i=1,2$) za realni točki. Z normalizacijo dobimo izotropne vektorje. %SLIKA
\seti
\end{enumerate}
\end{frame}
\begin{frame}
\frametitle{Algoritem}
\begin{enumerate}[1.]
\conti
\item Če nobena od možnih elips ne seka realno os na vsaki strani 0, potem preverimo, če to stori njihova skupna množica in ponovimo isti postopek.\medskip
\item Če ne najdemo take elipse niti za skupno množico, potem izračunamo še več lastnih vrednosti in lastnih vektorjev za $A(\theta)=\cos(\theta)H+\sin(\theta)iK$ za kote $\theta \not =0,\pi/2$ in delamo bisekcijo med točkami $rm$, $rM$, $im$, $iM$.\medskip
\item Končamo, ko najdemo definitno matriko $A(\theta)$ ali elipso, ki seka realno os na obeh straneh 0, nakar lahko uporabimo lemo \ref{komp}.
\end{enumerate}
\end{frame}

 \section{Numerična analiza}
 \begin{frame}[fragile]
 \frametitle{Numerična analiza}
 Označimo algoritme z \verb+AlgM+, \verb+AlgC+ in \verb+AlgCPU+. Za vsak algoritem preverimo:
 \begin{itemize}
 \item v kolikšnem času je našel (ali ni našel) rešitev,
 \item koliko izračunov lastnih vrednosti in vektorjev je potreboval (označimo kot „Koraki"),
 \item kako velika je napaka $|b^\ast Ab|$ (je \verb+Inf+ če ni rešitve).
 \end{itemize}
 \end{frame}

 \begin{frame}[fragile]
 \frametitle{$A$ in $\mu$ kompleksna}
 $B=F + \imath M$
 $$\verb~A=B+(-3+5i)*ones(200)-(200+500i)*eye(200)~$$
\begin{table}
\caption{A kompleskna, $\mu = 12000+10000\imath$.}
\begin{tabular}{|l|c|c|c|}
\hline
Algoritem & Čas & Koraki & Napaka\\
\hline
\hline
\verb+AlgM+& 0.433484&2 &1.7053e-13 \\
\hline
\verb+AlgC+ &0.656863&7&9.0949e-12\\
\hline
\verb+AlgCPU+ &0.03295&4&1.8190e-12\\
\hline
\end{tabular}
%\label{t7}
\end{table}
\end{frame}

\begin{frame}[fragile]
 \frametitle{$A$ realna, $\mu$ kompleksen}
Poglejmo primer iz članka \cite{trije}, za matriko $$\verb-A = randn(10)+(3+3i)*ones(10)-.$$
\begin{table}
\caption{A realen, $\mu = 23.2 +20\imath$ in naš algoritem ne najde rešitve.}
\begin{tabular}{|l|c|c|c|}
\hline
Algoritem & Čas & Koraki & Napaka\\
\hline
\hline
\verb+AlgM+ &0.013259&2&Inf\\
\hline
\verb+AlgC+ &0.022176 &6&1.1235e-14\\
\hline
\verb+AlgCPU+ &0.001088&3&1.7764e-15\\
\hline
\end{tabular}
%\label{t7}
\end{table}
\end{frame}

\begin{frame}
\frametitle{Bisekcija kota}
\begin{itemize}
\item Vsa presečišča elipse z realno osjo negativna in je $\Im (rM) >0$ $\Rightarrow$  bisekcija kota $3/2 \pi$ in $2\pi$ .
\item Izračunamo največjo lastno vrednost $A(7/4 \pi)$ ter pripadajoč lastni vektor $x_{nov}$. 
\item $p = x_{nov}^\ast A x_{nov}$ je točka na robu numeričnega zaklada med točkama $im$ in $rM$. 
\item Če je $\Im (p) <0$, potem iščemo presečišča $p$-$rM$ elipse z realno osjo, če pa je $\Im (p) >0$, potem iščemo presečišča $im$-$p$ elipse z realno osjo.
\item Ponavljamo, dokler ne najdemo presečišči elipse z realno osjo na obeh straneh 0, ali dokler $A(\theta)$ ne postane definitna.
\end{itemize}
\end{frame}
\begin{frame}
\begin{center}
\includegraphics<1>[scale=0.5]{prva.jpg}
\includegraphics<2>[scale=0.5]{prva-zoom.jpg}
\includegraphics<3>[scale=0.5]{druga-zoom.jpg}
\includegraphics<4>[scale=0.5]{druga-konec.jpg}
\end{center}

\end{frame}

\begin{frame}
\frametitle{Zaključek}
Naš boljši, če samo en korak. Carden hitrejši za manjše matrike. CPU boljši od vseh.
\end{frame}

\section{Literatura}
\begin{frame}[allowframebreaks]
\frametitle{Literatura}
\begin{thebibliography}{99}



\bibitem{carden}
R. Carden, \emph{A simple algorithm for the inverse field of values problem}, Inverse Probl. {\bf 25} (2009) 1--9.

\bibitem{carden_alg}
R. Carden, \emph{\texttt{inversefov.m}}, verzija 6.~2.~2011, [ogled 5.~5.~2016], dostopno na \url{http://www.caam.rice.edu/tech_reports/2009/}.

\bibitem{trije}
C. Chorianopoulos, P. Psarrakos in F. Uhlig, \emph{A method for the inverse numerical range problem}, Electron. J. Linear Algebra {\bf 20} (2010) 198--206.

\bibitem{lipkin}
N. Ciblak, H. Lipkin, \emph{Orthonormal isotropic vector bases}, v: Proceedings of DETC'98, 1998 ASME Design Engineering Technical Conferences (1998).

\bibitem{fovals}
W. Henao, \emph{\texttt{fovals.m}}, [ogled 27.~7.~2016], dostopno na \url{https://www.mathworks.com/matlabcentral/mlc-downloads/downloads/submissions/4679/versions/1/previews/fovals.m/index.html?access_key=}.

\bibitem{num}
Johnson, C. R., \emph{Numerical determination of the field of values of a general complex matrix}, SIAM J. Numer. Anal. {\bf 15} (1978) 595--602.

\bibitem{meurant}
G. Meurant, \emph{The computation of isotropic vectors}, Numer. Alg. {\bf 60} (2012) 193--204.

\bibitem{zaloga}
P. Psarrakos, M. Tsatsomeros, \emph{Numerical range: (in) a matrix nutshell}, Math. Notes from Washington State University,  Vol {\bf 45}, No. 2 (2002) 45--57.

\bibitem{nr}
P. Psarrakos, \emph{\texttt{nr.m}}, [ogled 12.~6.~2017], dostopno na \url{http://www.math.wsu.edu/faculty/tsat/files/matlab/nr.m}.

\bibitem{trije_alg}
F. Uhlig, \emph{\texttt{invfovCPU.m}}, verzija 22.~3.~2011, [ogled 5.~5.~2016], dostopno na \url{http://www.auburn.edu/~uhligfd/m_files/invfovCPU.m}.

\end{thebibliography}
\end{frame}

% \end{frame}
% \section{Literatura}
% \begin{frame}
% \frametitle{Literatura}
% \begin{enumerate}
% \item
% R. Carden, \emph{A simple algorithm for the inverse field of values problem}, Inverse Probl. {\bf 25} (2009) 1--9.
% \item
% R. Carden, \emph{\texttt{inversefov.m}}, verzija 6.~2.~2011, [ogled 5.~5.~2016], dostopno na \url{http://www.caam.rice.edu/tech_reports/2009/}.
% \item
% C. Chorianopoulos, P. Psarrakos in F. Uhlig, \emph{A method for the inverse numerical range problem}, Electron. J. Linear Algebra {\bf 20} (2010) 198--206.
% \item
% N. Ciblak, H. Lipkin, \emph{Orthonormal isotropic vector bases}, In: Proceedings of DETC'98, 1998 ASME Design Engineering Technical Conferences (1998).
% \seti
% \end{enumerate}
% \end{frame}
% \begin{frame}
% \begin{enumerate}
% \conti
% \item
% W. Henao, \emph{\texttt{fovals.m}}, [ogled 27.~7.~2016], dostopno na \url{https://www.mathworks.com/matlabcentral/mlc-downloads/downloads/submissions/4679/versions/1/previews/fovals.m/index.html?access_key=}.
% \item 
% Johnson, C. R., \emph{Numerical determination of the field of values of a general complex matrix}, SIAM J. Numer. Anal. {\bf 15} (1978) 595--602.
% \item 
% G. Meurant, \emph{The computation of isotropic vectors}, Numer. Alg. {\bf 60} (2012) 193--204.
% \item 
% P. Psarrakos, M. Tsatsomeros, \emph{Numerical range: (in) a matrix nutshell}, Math. Notes from Washington State University,  Vol {\bf 45}, No. 2 (2002) 45--57.
% \item
% P. Psarrakos, \emph{\texttt{nr.m}}, [ogled 12.~6.~2017], dostopno na \url{http://www.math.wsu.edu/faculty/tsat/files/matlab/nr.m}.
% \item
% F. Uhlig, \emph{\texttt{invfovCPU.m}}, verzija 22.~3.~2011, [ogled 5.~5.~2016], dostopno na \url{http://www.auburn.edu/~uhligfd/m_files/invfovCPU.m}.
% \end{enumerate}
% \end{frame}
\end{document}

