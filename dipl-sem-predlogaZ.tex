\documentclass[12pt,a4paper]{amsart}
\usepackage[slovene]{babel}
%\usepackage[cp1250]{inputenc}
%\usepackage[T1]{fontenc}
\usepackage[utf8]{inputenc}
\usepackage{amsmath,amssymb,amsfonts}
\usepackage{url}
%\usepackage[normalem]{ulem}
\usepackage[dvipsnames,usenames]{color}
\textwidth 15cm
\textheight 24cm
\oddsidemargin.5cm
\evensidemargin.5cm
\topmargin-5mm
\addtolength{\footskip}{10pt}
\pagestyle{plain}

\overfullrule=15pt % oznaci predogo vrstico

\theoremstyle{definition}
\newtheorem{definicija}{Definicija}[section]
\newtheorem{primer}[definicija]{Primer}
\newtheorem{opomba}[definicija]{Opomba}


\theoremstyle{plain}
\newtheorem{lema}[definicija]{Lema}
\newtheorem{izrek}[definicija]{Izrek}
\newtheorem{trditev}[definicija]{Trditev}
\newtheorem{posledica}[definicija]{Posledica}


\newcommand{\R}{\mathbb R}
\newcommand{\N}{\mathbb N}
\newcommand{\Z}{\mathbb Z}
\newcommand{\C}{\mathbb C}
\newcommand{\Q}{\mathbb Q}


% vstavi svoje definicije ...




\begin{document}


\thispagestyle{empty}
\noindent{\large
UNIVERZA V LJUBLJANI\\[1mm]
FAKULTETA ZA MATEMATIKO IN FIZIKO\\[5mm]
%\textcolor{Red}
{Finančna matematika} -- 1.~stopnja}
\vfill

\begin{center}{\large
Mirjam Pergar\\[2mm]
{\bf Računanje izotropnih vektorjev}\\[10mm]
Delo diplomskega seminarja\\[1cm]
Mentor: izred. prof. dr. Bor Plestenjak}
\end{center}
\vfill

\noindent{\large
Ljubljana, 2016} %letnica diplome
\pagebreak

\thispagestyle{empty}
\tableofcontents
\pagebreak

\thispagestyle{empty}
\begin{center}
{\bf Računanje izotropnih vektorjev}\\[3mm]
{\sc Povzetek}
\end{center}
% tekst povzetka v slovenscini

\vfill
\begin{center}
{\bf The computation of isotropic vectors}\\[3mm]
{\sc Abstract}
\end{center}
% tekst povzetka v anglescini

\vfill\noindent
{\bf Math. Subj. Class. (2010):}   \\[1mm]
{\bf Ključne besede:}   \\[1mm]
{\bf Keywords:}
\pagebreak


% tu se zacne tekst seminarja
\section{Uvod}

\textcolor{Red}{\begin{definicija}
Funkcija $f\colon [a,b]\to\R$ je {\em zvezna},če... 
\end{definicija}
%
Osnovne rezultate o zveznih funkcijah najdemo v \cite{glob}. Navedimo le naslednji izrek.
%
\begin{izrek}
Zvezna funkcija na zaprtem intervalu je enakomerno zvezna.
\end{izrek}
%
\proof
Izberimo $\varepsilon>0$. Če $f$ ni enakomerno zvezna, potem za vsak $\delta>0$ obstajata $x,y$, ki zadoščata
\begin{equation}\label{eq:razlika}
|x-y|<\delta\quad \text{in}\quad |f(x)-f(y)| \ge \varepsilon.
\end{equation}
\endproof
%
Oglejmo si še enkrat neenačbi \eqref{eq:razlika}.}

\begin{proof}
Here is my proof
\end{proof}

\subsection{\textcolor{Red}{Naslov morebitnega podrazdelka}}

\textcolor{Red}{\begin{lema}
Naj bo $f$ zvezna in ...
\end{lema}
}
\section{Realne matrike}
\section{Kompleksne matrike}
\section{Algoritmi}
\section{Zaključek}
\vfill

\textcolor{Red}{Primeri navajanja literature so razširjeni; najprej je opisano pravilo za vsak tip vira, nato so podani primeri. Posebej opozarjam, da spletni viri uporabljajo paket url, ki je vključen v preambuli. Polje ``ogled'' pri spletnih virih je obvezno; če je kak podatek neznan, ustrezno ``polje'' seveda izpustimo.}

\vfill

% seznam uporabljene literature
% \cite{meurant} za referenco
\begin{thebibliography}{99}

\bibitem{meurant}
G. Meurant, \emph{The computation of isotropic vectors}, Numer. Alg. {\bf 60} (2012) 193--204.

\bibitem{carden}
R. Carden, \emph{A simple algorithm for the inverse field of values problem}, Inverse Probl. {\bf 25} (2009) 1--9

\bibitem{trije}
C. Chorianopoulos, P. Psarrakos in F. Uhlig, \emph{A method for the inverse numerical range problem}, Electron. J. Linear Algebra {\bf 20} (2010) 198--206

\bibitem{referenca-clanek}
\textcolor{Red}{I.~Priimek, \emph{Naslov članka}, okrajšano ime revije {\bf letnik revije} (leto izida) strani od--do.}

\bibitem{navodilaOMF}
\textcolor{Red}{C.~Velkovrh, \emph{Nekaj navodil avtorjem za pripravo rokopisa}, Obzornik mat.\ fiz.\ {\bf 21} (1974) 62--64.}

\bibitem{vec-avtorjev}
\textcolor{Red}{P.~Angelini, F.~Frati in M.~Kaufmann, \emph{Straight-line rectangular drawings of clustered graphs}, Discrete Comput.\ Geom.\ {\bf 45} (2011) 88--140.}

\bibitem{referenca-knjiga}
\textcolor{Red}{I.~Priimek, \emph{Naslov knjige}, morebitni naslov zbirke  {\bf zaporedna "stevilka}, zalo"zba, kraj, leto izdaje.}

\bibitem{glob}
\textcolor{Red}{J.~Globevnik in M.~Brojan, \emph{Analiza I}, DMFA - zalo"zni"stvo, Ljubljana, 2010.}

\bibitem{lang}
\textcolor{Red}{S.~Lang, \emph{Fundamentals of differential geometry}, Graduate Texts in Mathematics {\bf 191}, Springer-Verlag, New York, 1999. }

\bibitem{referenca-spletni-vir}
\textcolor{Red}{I.~Priimek, \emph{Naslov spletnega vira}, verzija "stevilka/datum, [ogled datum], dostopno na \url{spletni.naslov}.}

\bibitem{glob-splet}
\textcolor{Red}{J.~Globevnik in M.~Brojan, \emph{Analiza 1}, verzija 15.~9.~2010, [ogled 12.~5.~2011], dostopno na \url{http://www.fmf.uni-lj.si/~globevnik/skripta.pdf}.}

\bibitem{wiki}
\textcolor{Red}{\emph{Matrix (mathematics)}, [ogled 12.~5.~2011], dostopno na \url{http://en.wikipedia.org/wiki/Matrix_(mathematics)}.}

\end{thebibliography}

\end{document}

